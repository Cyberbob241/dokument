\chapter{Evaluation}

	Auf Basis der im Rahmen dieser Arbeit erlangten Erkenntnisse wird im folgenden Kapitel die Erfolgswahrscheinlichkeit der konzipierten Optimierungsma�nahmen evaluiert, wobei insbesondere deren Realisierbarkeit \mbox{in produktiven Analyseumgebungen analysiert wird.}

	\section{Erfolgswahrscheinlichkeit der Detektionmethoden}\label{sec:wkt}
	
		Basierend auf den Resultaten der in Kapitel \ref{cha:Virtualisierung} erarbeiteten Verfahren muss die Realisierung einer exhaustive Detektionspr�vention aufgrund der hohen Diversit�t und Heterogenit�t potentieller Virtualisierungs-Indikatoren als kaum umsetzbar konstatiert werden. Insbesondere bei der kombinierten Verwendung verschiedener Detektionstechniken k�nnen belastbare Informationen �ber die eingesetzte Systemumgebung deduziert werden, was final in der Identifikation von pr�senten Hypervisor-Realisierungen resultiert. Aus der Perspektive des Malware-Autors wird die Implementierung von Detektionsmethoden lediglich durch die intentionale Minimierung des quantitativen Malware-Profiles limitiert. Um die weitestgehende Verdecktheit bei Ausf�hrung und Distribution der Binaries garantieren zu k�nnen, werden h�ufig besonders erfolgversprechende Detektionsvektoren mit vergleichsweise geringem Implementierungsaufwand kombiniert.
		Triviale Methoden zur Identifikation virtueller Systeme, welche prim�r auf der Examination leicht manipulierbarer Systemparameter wie MAC-Adressen basieren, werden innerhalb dedizierter Analyseumgebungen unter Verwendung entsprechender Systemkonfigurationen meist automatisiert oder durch manuelle Rekonfiguration verhindert. 
		Die Malware-Binaries implementieren daher h�ufig ausgew�hlte Spezialf�lle, wobei die Integrit�t der im Detektionskontext instrumentalisierten Attribute  oftmals essentiell f�r die korrekte Funktionalit�t der virtualisierten Systemumgebung ist. Auf diese Weise k�nnen manuelle Modifikationen, beispielsweise durch einen Analysten, weitestgehend ausgeschlossen werden. Exemplarisch sei an dieser Stelle auf die Treibersignaturen emulierter Hardware-Komponenten verwiesen, welche h�ufig kompromittierende Namensattribute enthalten. Da die emulierte Treiberfunktionalit�t zur Interaktion mit der verwendeten Hypervisor-L�sung jedoch obligatorisch ist, erweisen sich umfangreiche Modifikationen zur Verschleierung des virtuellen Charakters aufgrund der systeminternen Komplexit�t h�ufig als kaum realisierbar.\\
		\noindent Unter Ber�cksichtigung der zunehmenden Expansion in der produktiven Nutzung von virtualisierten Systemen verzichten einige Malware-Implementierungen bewusst auf die Detektion des virtuellen Systemkontextes, um den Umfang potentiell relevanter Zielsysteme nicht zu limitieren \textcolor{red}{[11]}. 
		%Quelle: https://books.google.de/books?id=hrSoCAAAQBAJ&pg=PA17&lpg=PA17&dq=less+vm+detection&source=bl&ots=KqJDeb-Nho&sig=YFe6waQMlr4Qhd1gvDu72OlBQBY&hl=de&sa=X&ved=0ahUKEwjP-rOZ5OzZAhXMjSwKHdzNCUkQ6AEIUTAF#v=onepage&q=gain%20popul&f=false 
		Um dennoch die Examination der Malware innerhalb dedizierter Analyseumgebungen obstruieren zu k�nnen, wird anstelle der Virtualisierungs-Parameter die Detektion von Indikatoren der stark spezialisierten Analyseumgebungen forciert. Dieser Prozess kann beispielsweise durch die gezielte Identifikation von speziellen Debugger- und Analyseprozessen sowie kompromittierenden Namensattributen einzelner Systemdateien realisiert werden. Die verwendeten Konzepte sind dabei grundlegend kongruent zum Vorgehen bei der Erkennung virtueller Systeme, weshalb auch in diesem \makebox[\linewidth][s]{Kontext eine potentiell hohe Detektionswahrscheinlichkeit konstatiert werden muss.}
		
	
	\section{Realisierbarkeit der Optimierungsans�tze}
	
		\textcolor{red}{Ostendorff}
	
		- Aleatorische Evasionstechniken von Malware (Click-Trigger, Reboot-Trigger) kaum deterministisch emulierbar\\
	
		\noindent - Komplexit�t von tiefgreifenden Modifikationen wie Entropie-Manipulation\\
		
		\noindent - Verfahren liefern theoretische Ans�tze, hohe Komplexit�t in \\
		  realit�tsnahen Anwendungsszenarien\\
		  
		\noindent Modifikationen der Analyseumgebungen induzieren wiederum Anomalien, die durch Malware-Implementierung detektierbar. Exemplarisch systeminterne Reduktion Entropie auf Minimalsatz referenziert, suboptimale Realisierung erm�glicht Detektion virtualisierter Analyseumgebung durch Testroutinen innerhalb der Malware. \\\\
		  
		 \noindent ==> hoher Aufwand potentiell tiefgreifender Systemmodikfikationen , exhaustive Pr�vention nicht m�glich, kontinuierliche Adaption neuer Detektionstechniken erforderlich, induzieren neue Ans�tze zur Realisierung Detektionsmechanismen \textcolor{blue}{<TXT>}
		 
		 
		 Insbesondere Automatisierung oftmals kritisch. 